\chapter{Introduction}
\label{chap:introduction}

\section{Project Description}
\subsection{Background}
 - Made a couple of smaller games in other courses. 
 - Wish to try and develop a more complex networked game for the bachelor project.
 - Wish to challenge ourselves by making a game with many different abilities and components as well as trying to keep them somewhat balanced 
 - No plans for commercial release

We are three students who have worked together on several projects before during our bachelor program. We wanted to make a Twin Stick MOBA type game for PC, intended to be played with a game controller. Our wish is to further expand our knowledge of game programming with focus on networking and game balancing. Figuring out good practices for local responsiveness coupled with consistent networked behaviour and how to implement these is something we would like to learn. This is good knowledge to have in a world where multiplayer, and the ability to stay connected is very important, even in games. Our game is based around having many different types of "Docking Kits" with four different abilities each, letting us experience what it's like to come up with and balance large amount of components.

Having worked with Unity 5.x together on previous projects, this felt like the natural choice for our development environment. Unity is also a very popular engine used worldwide \cite{unityUsageStatistics}, potentially leaving the knowledge we gain about the engine very valuable. Since we've chosen Unity, our networking will therefore be the Unity Networking and their "high-level" scripting API (HLAPI) \cite{unityUNETManual} and the advantages / challenges this provides.

As a group we consider this as a project for learning, and we have no plans of making a commercial release for our game. 

\subsection{Goals}
One of our primary goals for the project is to acquire more experience with larger game projects as this is our first attempt at such a large task. We wish to further our skills and understanding of Unity with a particular focus on how to implement networked functionality. 
In regards to the networking we want to learn good practices for local responsiveness that still allows us to handle important verifications on the server side. 

We wish attain more experience with asymmetric balancing on a larger scale by designing and implementing a large variety of docking kits, each with their own tools and abilities. 

We want to learn more about professional tools like Jira and Confluence as well as how we can use these in the development of the project to provide a robust workflow. As part of learning to use these professional tools we also want to improve our skills at estimating the time needed to implement features. 

In the end we would like to have created a game that features robust networked functionality and an extensible framework that is easy to improve and add new components to. 

\section{Academic Background}
We
 - Andreas:

 - Martin:

 - Sondre:

\section{Project Audience}
The thesis is written for fellow students and other game programmers who are planning to work with Unity and networking in particular. The contents should allow the reader to learn more about technical details of working with Unity and our ways of solving the various challenges that might show up when working in the engine. 
We expect the reader to have some knowledge with C\# and Unity, but will detail some core concepts wherever we feel necessary. 
  
\section{Thesis Structure}
The thesis is divided up into nine different chapters with appendices at the end. The chapters of the thesis include:
\begin{enumerate}
    \item \hyperref[chap:introduction]{Introduction}: Introduction to the thesis and its contents.
    \item \hyperref[chap:gamedesign]{Game Design}: The initial game design and how it changed. 
    \item \hyperref[chap:technical]{Technical Design}: How the game is architectured and how the various components work on a techincal level.
    \item \hyperref[chap:process]{Development Process}: The tools and software development model that was used. 
    \item \hyperref[chap:implementation]{Implementation}: Specifics on the challenges that had to be overcome during development and their implementation.
    \item \hyperref[chap:deployment]{Deployment}: How the game is packaged and distributed.
    \item \hyperref[chap:testing]{Testing and User Feedback}: Details the testing process and the feedback gotten from user testing.  
    \item \hyperref[chap:discussion]{Discussion}: Discusses the results of the project and the development decisions that were made.
    \item \hyperref[chap:conclusion]{Conclusion}: Reflects on the finished project. 
\end{enumerate}
    